\subsection{Limitations of CluSanT}

Our experimental evaluation of CluSanT reveals several fundamental limitations that affect its applicability to general PII protection tasks. While CluSanT demonstrates technical sophistication with its multi-word embedding capabilities and differential privacy guarantees, our results show only 15.8\% overall PII protection, significantly lower than other mechanisms (InferDPT: 99.6\%, PhraseDP: 98.9\%, SANTEXT+: 83.9\%).

\subsubsection{Fundamental Incompatibility with PII Protection}
The primary limitation of CluSanT is its fundamental incompatibility with PII protection due to the inherently unpredictable and infinite nature of personally identifiable information. PII cannot be pre-defined because names, emails, and addresses have unlimited combinations, new domains and naming patterns emerge constantly, and PII protection must work on any sensitive information, not just pre-known tokens.

This creates a privacy paradox: to protect sensitive information, CluSanT must first know what information is sensitive, but PII is inherently unpredictable. Our experimental results demonstrate this limitation clearly: CluSanT achieves 0\% name protection because common names like "Aaliyah Popova" and "Konstantin Becker" are not present in its embedding vocabulary, and the framework falls back to no sanitization for unknown tokens.

The approach is fundamentally flawed for PII protection because it attempts to solve an infinite, dynamic problem (unpredictable PII) with a finite, static solution (pre-defined token sets).

\subsubsection{Domain-Specific Training Bias}
CluSanT's embedding file is constructed from the TAB dataset (European Court of Human Rights legal documents), augmented with GPT-4o to generate 100 similar tokens per original token. This domain-specific training creates a vocabulary biased toward legal and political entities (e.g., "Sinn Fein headquarters" $\rightarrow$ "Labour Party headquarters"), which does not generalize to general PII protection tasks. The 60\% address protection rate reflects some overlap with European legal addresses, while the 10\% email and 30\% phone protection rates indicate poor coverage of personal contact information patterns.

\subsubsection{Limited Scalability and Transfer Learning}
Adapting CluSanT to new domains requires complete retraining, including token extraction, GPT-4o augmentation, embedding generation, and clustering. This process is computationally expensive and time-consuming, with no transfer learning capability between domains. The framework's consistency across epsilon values (15.8\% protection for all $\epsilon \in \{1.0, 1.5, 2.0, 2.5, 3.0\}$) indicates that the poor performance stems from vocabulary mismatch rather than parameter tuning issues.

\subsubsection{Comparison with Alternative Approaches}
Unlike pattern-based approaches (regex) that can handle infinite variations of PII patterns without pre-defined token sets, or LLM-based approaches (SANTEXT+) that can dynamically identify PII, CluSanT's embedding-based approach fundamentally requires knowledge of sensitive tokens beforehand. This creates a privacy paradox: to protect sensitive information, the system must first know what information is sensitive, but PII is inherently unpredictable.

Pattern-based approaches work because they recognize structural patterns (email format, phone number format) rather than specific token values. LLM-based approaches work because they can dynamically identify sensitive entities using contextual understanding. CluSanT's approach fails because it attempts to solve an infinite, dynamic problem (unpredictable PII) with a finite, static solution (pre-defined token sets).

Our analysis suggests that CluSanT is well-suited for bounded, predictable domains such as legal document anonymization where sensitive tokens can be exhaustively enumerated, but is fundamentally incompatible with the unpredictable nature of real-world personally identifiable information. The framework's technical advantages (multi-word embedding capability, differential privacy guarantees) are undermined by this fundamental architectural limitation in general PII protection scenarios.
